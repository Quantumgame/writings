\chapter{Discussion}

In this work we have shown that Zebra finch syllables can be quantified by their acoustic features in a duration-independent fashion, that some of these acoustic features, mainly amplitude, spectral distribution statistics, pitch saliency, and temporal entropy, can be used to predict both spike rate and LFP power. We showed that these features can be decoded from the spike rate vector of a population of neurons, and that the decoding performance grows as more neurons are utilized. We showed that the power spectrum of the LFP in the Zebra finch auditory system contains a significant amount of information about the acoustic features of vocalizations, and that regional differences exist in the type of information decoded.
    Training encoders enabled us to say what causally drove spike rate or LFP power on individual neurons or electrodes \cite{Weichwald2015}, while decoders enabled us to determine the types of information that could be successfully extracted from population activity (Figure 6). The acoustic features that drove neurons the most, such as maximum amplitude and spectral distribution statistics, were also acoustic features that were well decoded. However, there was a discrepancy for temporal entropy. Neuron spike rate and LFP power were often tuned to temporal entropy and encoders performed relatively well (Figure 5b), but it was not one of the top decoded acoustic features (Figure 6a). A similar situation existed for the 2nd voice acoustic feature, which drove high frequencies relatively well, but was very poorly decoded.
    We found that decoding from the LFP power spectra outperformed decoding from the population spike rate vector. This is not necessarily surprising, as the spike rate was computed for syllables of varying duration, and spike timing is likely to play a significant role in encoding time-varying spectro-temporal information. However, we found that including pairwise synchrony terms with the population spike rate vector decoder boosted performance to that of the LFP power spectra. We think that by including the synchrony terms, which are effectively the normalized dot product between two binned spike trains, we are adding back temporal information that was lost when averaging spikes over time to produce rates.
    Region L3 was found to be best predicted by the univariate tuning curves among other regions, while containing the most information about spectral distribution statistics and the time-varying fundamental frequency, and being more invariant to amplitude than CM, L1, or L2. Other researchers have found that region L3 was found to be more selective and tolerant for small changes in acoustic features \cite{Meliza2012}, better at discriminating distance calls \cite{Elie2015a}, and overall more sparsely firing, noise correlated, and selective \cite{Calabrese2015}. Our results complement the findings of selectivity by these researchers; a neural population must adequately represent the features of a vocalization to be selective. We found L3 to have a higher invariance to amplitude, which could contribute to its tolerance to acoustic perturbations in vocalizations. However, it is somewhat surprising that a region found to be tolerant of perturbations in acoustic features contains so much information about the statistics of the spectral distribution. Perhaps this information is encoded by inhibitory neurons and used to suppress the firing of excitatory neurons, endowing them with their selectivity.
    Finally, we demonstrated a concrete relationship between the local spike rate and spike synchrony, and LFP power that the population produces. If the LFP is comprised predominantly of synaptic currents, then we are showing that those synaptic currents are directly translated into the average spike rates of neurons, and enabled us to predict the LFP power from spike rate. We found that the addition of spike synchrony boosted our predictive power for frequencies above 50Hz. Synchronous synaptic currents are thought to be a significant contributor to LFP power \cite{REF}. Taken together with the results that decoder performance of the LFP power spectrum is higher than the population spike rate vector, and that computing the power spectrum of a raw LFP is computationally cheaper than online spike sorting, we have demonstrated a decoding methodology that could improve the performance of brain-machine interfaces.

