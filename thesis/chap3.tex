\chapter{The Relationship between Spikes and LFP}

\section{LFP Power is a Mix of Local Population Spike Rate and Synchrony}

    We attempted to quantify the relationship between population spike rate, spike synchrony, and the LFP as a function of frequency. To do this, we built an encoder that attempted to predict power at a given frequency band for a given electrode from the population spike rate, and in addition spike synchrony (see Methods - Population Spike Rate and Spike Synchrony and Methods - Spike Rate to LFP Power Encoder).
    Figure 8a shows that on average, the LFP power can be well predicted by population spike rate, and adding spike synchrony terms boost performance for frequency bands from 45-165Hz. The figure also shows variation in performance by frequency. To quantify this, and to explore region specificity, we fit a linear model in R to predict encoder performance from anatomical region, plus the interaction of frequency and population representation type (“Spike Rate” or “Spike Rate + Synchrony”). The effect sizes for the interaction terms in the model are shown in Table 4, effect sizes for regions are provided in the figure legend. LFP power was best predicted for Field L regions, and was not as well predicted for CM and NCM.
    The LFP is typically described as the summed local electrical activity on an electrode. We investigated how much of an effect neurons had on LFP power as a function of distance from the electrode. To do this, we utilized the weights of the encoder trained to predict LFP power from spike rate. Each neuron, with its associated decoder weight, was a given distance from the electrode whose LFP was being predicted. Figure 8b shows smoothed squared encoder weights as a function of distance. The weight-squared are shown for the 33-49Hz band and the 165-182Hz band. Weights-squared for the higher frequency band fall off quickly with distance, while they fall off less sharply for 33Hz, and even begin to increase at long range distances. The inset of Figure 8b shows the average encoder weights-squared for neurons on the same electrode, vs neurons on a different electrode. Neurons on the same electrode contribute much more to LFP power, an order of magnitude more, than neurons on other electrodes. From this, we conclude that the LFP power is comprised predominantly of local spike rate and synchrony.

