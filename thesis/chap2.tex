\chapter{Decoding Acoustic Features from the Neural Ensemble}

\section{Regional Specificity in Decoding Performance}

    The Zebra finch auditory system is not a homogenous structure. There is some evidence that it is anatomically layered in a way homologous to mammalian cortex [Wang2010]. A detailed analysis of regional specificity by [Meliza2012] showed that regions L2 and L1 were the least selective and tolerant, responding to most acoustic stimuli in a way that is not invariant to slight changes in acoustic features, while regions NCM and L3 were the most selective and tolerant. [Elie2015a] analyzed the decoding performance of call type using the same dataset analyzed in this work, and found regional differences as well. They found that regions L3 and CM were the best at classifying Distance Calls and Field L were the best at classifying song. They also found that regions L3 and CMM was the most invariant (or tolerant) of variation in Distance calls.
    The panels in Figure 7a show single electrode decoder performance across space, for all electrodes across recording sites, with electrodes from the two hemispheres plotted together as a function of their distance from the midline along the medial-lateral axis, and their rostral-caudal distance from region L2A. Figure 7b shows single electrode decoder performance averaged within acoustic feature and region. A linear model was fit in R to predict single electrode decoder performance from the interaction between acoustic feature and region. Table 3 shows the effect sizes of the interactions for this model. The panels in Figure 7a show that maximum amplitude has some region specificity, with electrodes in region L2 containing the most amplitude information. Mean spectral frequency also exhibited some regional specificity, being decoded best from regions L2 and L3. The coefficient of variation for the fundamental frequency (CV F0), which would be low for syllables that do not vary much spectrotemporally, like female distance calls, was also decoded best from region L3. Saliency and temporal entropy do not exhibit much regional specificity. Further examination of Table 3 shows that decoders trained on electrodes in region L3 outperform other regions for most spectral features.

\section{Ensemble Decoding Performs Best for Amplitude and Spectral Features}

   We built decoders to predict each individual acoustic feature from population activity, encoded by the population spike rate vector, LFP power spectra, or population spike rate vector and in addition, pairwise spike synchrony (Methods - Encoder and Decoder Dataset Construction). Figure 6a shows the mean performance by neural response type and acoustic feature. Visualization makes it clear that maximum amplitude is decoded best from the population data, and also that LFP power spectra outperform population spike rate in decoding performance. Notably, the addition of pairwise synchrony terms to the population spike rate boosts decoder performance to that of the LFP power spectra. We fit a linear model in R to determine the predictability of an acoustic feature as a function of the interaction between acoustic feature and the neural representation used. Table 2 shows the effect size of the interaction terms, demonstrating quantitatively that amplitude and spectral features are best decoded from the population data.
   We further investigated how decoder performance increased as a function of number of electrodes. To do this, we first merged electrodes from each hemisphere for each recording site, giving us potentially 32 electrodes to predict from. Then we ran a decoder for a variety of combinations of electrodes for a fixed number of electrodes (Methods - Ensemble Decoding Analysis). The results are shown in Figure 7b and 7c for the best decoded acoustic features. Common trends occur for across features. When decoding from population spike rate, decoder performance increased sharply from 1 to 10 neurons. For the LFP power spectra, decoder performance increased as sharply from one to four electrodes. After that point, the slope remained positive, and adding more neurons or electrodes gradually improved decoder performance. For each site, maximum decoder performance was reached using all neurons or electrodes. We conclude that there is much more information in multi-electrode representations than single electrodes.

